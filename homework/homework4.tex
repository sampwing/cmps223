\documentclass[11pt]{amsart}
\usepackage{geometry}                % See geometry.pdf to learn the layout options. There are lots.
\geometry{letterpaper}                   % ... or a4paper or a5paper or ... 
%\geometry{landscape}                % Activate for for rotated page geometry
%\usepackage[parfill]{parskip}    % Activate to begin paragraphs with an empty line rather than an indent
\usepackage{graphicx}
\usepackage{amssymb}
\usepackage{epstopdf}
\usepackage{enumerate}
\DeclareGraphicsRule{.tif}{png}{.png}{`convert #1 `dirname #1`/`basename #1 .tif`.png}

\title{CMPS223: Homework \#4}
\author{Sam Wing}
%\date{}                                           % Activate to display a given date or no date

\begin{document}
\maketitle
\section{MicroIL}

Below are the given examples walking them through using MicroIL:

\begin{enumerate}[a)]
	\item push0 $\cdot$ inc $\cdot$ halt
	\begin{enumerate}[1)]
		\item P $\vdash$ $< 1, f, s >$
		\item $\rightarrow$ $< 2, f, 0 \cdot s >$ (push0)
		\item $\rightarrow$ $< 3, f, 1 \cdot s >$ (inc)
		\item (halt)		
	\end{enumerate}
	
	\item inc $\cdot$ inc $\cdot$ halt
	\begin{enumerate}[1)]
		\item P $\vdash$ $< pc, f, s >$
		\item $\rightarrow$ $<pc + 1, f, (n + 1) \cdot s >$ (inc -error, cannot increment epsilon)
	\end{enumerate}
\end{enumerate}

Although the first program (a) completes successful, the second (b) does not.  This is because before program$_b$ has any value on the stack, it tries to increment the value.  We cannot allow the program to increment $\epsilon$ because there is nothing on the stack yet, so I believe that this program will fail.  

Below is the type during each function call throughout program$_a$:

\begin{enumerate}[a)]
	\item push0 $\cdot$ inc $\cdot$ halt
	\begin{enumerate}[1)]
		\item $P_1$ = init, $F_1$ = Top, $S_1$ = $\epsilon$	
		\item $P_2$ = push0, $F_2$ = Top, $S_2$ = Int $\cdot $  $\epsilon$
		\item $P_3$ = init, $F_3$ = Top, $S_3$ = Int $\cdot$ $\epsilon$
		\item F, S, 3 $\vdash$ P
	\end{enumerate}
\end{enumerate}

\section{Operator Equivalence}

\begin{verbatim}

function f():
    return random_int()
 
function g_x(a):
    return a & true

function g_y(a):
    return (a ? true : false) & true

\end{verbatim}

We are trying to show that the equivalence is not preserved through these two logically similar functions.  Say for example that we get a value b := f(), We will notice that for all positive and even numbers generated by this function that, $g_x$(b) $\rightarrow$ false, this is because true is internally stored as a 1 and the bitwise and operation between 1 and any odd number will be 0.  However $g_y$(b) $\rightarrow$ true.  This shows that when b is positive and even the equivalence between $g_x$ and $g_y$ is not held.  However when the value of b is odd the relation between the two functions will hold, due to the bitwise and operation, true(1) \& b $\rightarrow$ 1.


\section{Erlingsson's Defense Techniques}

After reading through the paper I believe that I know which of the defense techniques rely on the secrecy of certain information.  

Defense$_1$ is the first of which that comes to mind.  This defense is the use of stack canaries, in the place of more expensive encryption techniques applied to the program as it runs.  The canaries are placed above function-local stack buffers, they can take on either a public known value or a randomly generated cookie.  The canaries value is known within the program and is used to check whenever a call returns from the stack that no buffer overflow took place which would have modified the value of the canary.  This is useful for knowing whether there was a buffer overflow, however if the value of the canary is known then it is possible for a buffer overflow to modify data and rewrite the canary with its original value such that it appears that nothing went wrong.  

Defense$_5$ is another defense technique, which also employs the use of encryption.  In this method, addresses and pointers are encrypted throughout the program.  This can make programs which do not use this technique uniformly difficult or even unstable, as pointer arithmetic is fairly common place in low level programming.  However, if the encryption is used uniformly throughout the program then programmers can implement methods to still use pointer arithmetic as long as they take note of the encryption methods being used.  This defense technique can have a significant computational overhead for the encrypting/decrypting of these documents depending on the technique in place.  It appears that Windows Vista uses a simple xor encryption method which only creates a very small bottleneck.




\end{document}  