\documentclass[11pt]{amsart}
\usepackage{geometry}                % See geometry.pdf to learn the layout options. There are lots.
\geometry{letterpaper}                   % ... or a4paper or a5paper or ... 
%\geometry{landscape}                % Activate for for rotated page geometry
%\usepackage[parfill]{parskip}    % Activate to begin paragraphs with an empty line rather than an indent
\usepackage{graphicx}
\usepackage{amssymb}
\usepackage{epstopdf}
\usepackage{enumerate}
\DeclareGraphicsRule{.tif}{png}{.png}{`convert #1 `dirname #1`/`basename #1 .tif`.png}

\title{CMPS223: Homework \#3}
\author{Sam Wing}
%\date{}                                           % Activate to display a given date or no date

\begin{document}
\maketitle
\section{Access Control Matrix}

Consider a system with three users A, B, and C, two objects O and P,
and one operation on the objects.
Consider the following policy (written in Binder notation):
\begin{itemize}
	\item trusted(A).
	\item A says trusted(B).
	\item A says may-access(C,O).
	\item may-access(x,O) :- trusted(y), y says may-access(x,O).
	\item may-access(x,P) :- trusted(x).
	\item trusted(x) :- trusted(y), y says trusted(x).
\end{itemize}

We can reduce this as follows:
\begin{itemize}
	\item trusted(A) by trusted(A)
	\item trusted(B) by trusted(B) :- trusted(A), y says trusted(B)
	\item may-access(A,P) by may-access(A,P) :- trusted(A)
	\item may-access(B,P) by may-access(B,P) :- trusted(B)
	\item may-access(C,O) by may-access(C,O) :- trusted(A), A says may-access(C,O).
\end{itemize}

\begin{table}[h]
\centering
\begin{tabular}{| l | c | c | }
\hline
properties $\diagdown$ objects  & $O$ & $P$  \\ \hline
$A$ &  & r\\ \hline
$B$ &  & r \\ \hline
$C$ & r &  \\ \hline
\end{tabular}
\caption{Problem \#1 Access Control Matrix, r indicates that the property may-access the object}
\label{acl}
\end{table}

\section{Security and Relational Semantics}
The following questions are about the definition of security by Joshi and Leino. If in doubt, you may use the relational version of the definition. For (a)�(g), simply answer �yes�, �no�, or �it depends�.

\begin{enumerate}[a]
	\item Yes
	\item Yes
	\item No
	\item No
	\item Yes
	\item No
	\item It Depends
	\item We know that the program S is deterministic due to its ending condition being h,k = 0.  So we can use lemma 27 from the paper to prove that it is secure. \\ S is secure $\equiv$ ($\forall M$ :: wlp.S.(k $\neq$ M) $\epsilon$ Cyl)
	\item So (f) is not secure because either S or T are revealed depending on the polarity of k.  (g) can possibly be insecure because the composition can possibly reveal information about the secure variables, say S is an adds some value to 0, then S ; T will yield T.  But there are cases when the composition of S ; T will not reveal any information about (g).
\end{enumerate}


\section{HiStar}
Given that there exists a thread T and a segment S, and suppose that T can start another thread T' that can read S. Discuss whether it follows that T can also read S.  This implies $L_S \sqsubseteq_{O_{T'}} L_{T'}$.

\begin{enumerate}[a]
	\item No, $L_T$ does not necessarily equal $L_{T'}$, in the paper it is not specified whether new threads are granted the labels of their parent threads. So I assumed the threads had no privileges until they were otherwise granted.
	\item The paper states "Threads ... can be created with an ownership set O as long as $O \subseteq O_T$."  While it doesn't specify the default behavior of thread creation, we know that if $L_S \sqsubseteq_{O_{T'}} L_{T'}$ and that  $O_{T'} \subseteq O_T$, that $O_T$ contains \emph{at least} all the ownership properties of $O_{T'}$.  So making $O_T = O_T'$ doesn't change the result.
	\item Yes, in this case we already know that $O_T$ is a superset of $O_{T'}$, and now given that $L_T = L_{T'}$ is is true that  $L_S \sqsubseteq_{O_{T}} L_{T}$.
\end{enumerate}

\end{document}  